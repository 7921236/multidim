\documentclass{beamer}
\graphicspath{ {./images/} }

\mode<presentation> {
  \definecolor{frameheadforeground}{RGB}{169,33,62}
  \definecolor{frameheadbackground}{RGB}{255,255,255}

  \usetheme{Warsaw}
  %\setbeamercovered{transparent}
}

\setbeamercolor{structure}{fg=frameheadforeground,bg=frameheadbackground}

\usepackage[utf8]{inputenc}
\usepackage[MeX]{polski}

\begin{document}

\begin{frame}
\title[Tytuł]{Analiza wielowymiarowa}
\subtitle{Environment}
\author{Maciej Nasinski, Paweł Strawiński}
\institute{Uniwersytet Warszawski}
\date{Zajęcia 1 \\ 6 października 2022}
\titlepage
\end{frame}

\begin{frame}{Tools}
  \begin{itemize}
  \item STATA 17 \url{https://www.wne.uw.edu.pl/pl/wydzial/pracownia-informatyczna/}
  \item python \url{https://www.python.org/}
  \item pip or anaconda (e.g. python packages like jupyter/jupyterlab)
  \item git clone \url{https://github.com/polkas/multidim}
  \end{itemize}
\end{frame}

\begin{frame}{Environment}
  \begin{itemize}
    \item \texttt{terminal (zsh lub bash) and pip/conda.}
    \item \texttt{https://github.com/Polkas/multidim}
  \end{itemize}
\end{frame}

\begin{frame}{Update multidim}
  \begin{itemize}
    \item \texttt{cd multidim}
    \item \texttt{(optionally) git stash or git add . and git commit -am "opis"}
    \item \texttt{git pull}
  \end{itemize}
\end{frame}

\begin{frame}{STATA and python}
  \begin{itemize}
  \item pystata documentation \url{https://www.stata.com/python/pystata/index.html}
  \end{itemize}
\end{frame}

\begin{frame}{Better to ask the way than to go astray}
  \begin{itemize}
  \item stackoverflow and cross validated category \url{https://stackoverflow.com/}
  \end{itemize}
\end{frame}

\begin{frame}{Python - sources}
  \begin{itemize}
  \item Python for Everybody \url{https://py4e.pl/book.php} or Python Crash Course \url{https://github.com/ehmatthes/pcc_2e}
  \item \url{https://realpython.com/}
  \end{itemize}
\end{frame}

\end{document}
